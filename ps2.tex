%-- Generated using Rammy v0.1.0 | github.com/TimboKZ/Rammy
%-- Module: latex-common
%-- Template: problem-set

%-- summary: Template for lecture notes with a lot of useful snippets.

\documentclass{article}

\usepackage[table]{xcolor}

% Title for the `compact-header` snippet
\newcommand\HeaderSubject{CS 153~~~Communication Complexity}
\newcommand\HeaderAuthor{Timur Kuzhagaliyev}

%-- Rammy start ----------------
\input{latex-common/snippets/symbols.tex} %-- module-snippet: latex-common/symbols
\input{latex-common/snippets/problem-set-header.tex} %-- module-snippet: latex-common/problem-set-header
\input{latex-common/snippets/urls.tex} %-- module-snippet: latex-common/urls
\input{latex-common/snippets/figures.tex} %-- module-snippet: latex-common/figures
\input{latex-common/snippets/code.tex} %-- module-snippet: latex-common/code
\input{latex-common/snippets/misc.tex} %-- module-snippet: latex-common/misc
%-- Rammy end ------------------

% Don't wrap long matrices
\setcounter{MaxMatrixCols}{20}

\begin{document}
\problemset{Problem Set \#2 Solutions}{Out: May 22}{Due: \bf {June 1}}

\begin{enumerate}
    \item Solutions:  
    \begin{enumerate}
        \item We can show that the inequality in the question holds by
            considering the structure of $M_1$ and $M_2$. The breakdown of
            matrix of $M$ into $M_1$ and $M_2$looks like

            \begin{equation}\label{eq:appendrow}
                \left(\begin{array}{c>{\columncolor{olive!20}}cc}
                \x  & X_1 \x T & \x \\
               \rowcolor{blue!20}
               S \x Y_1 & S \x T & S \x Y_2 \\
                \x & X_2 \x T & \x \\
              \end{array}\right)
            \end{equation}

            where the highlighted row is matrix $M_1$, the highlighted column
            is the matrix $M_2$ and $X_1, S, X_2$ with $Y_1, T, Y_2$ are
            partitions of $X$ and $Y$ respectively. Note that due to the
            structure of given matrices, the standard rank of $M_1$ is bound
            above by $|S|$ and the standard rank of $M_2$ is bound above by
            $|T|$, and the same bounds apply to $triangle-rank$.
            \\\\
            Note also that $S \x T$ is a zero matrix - this means that
            regardless of how rows or columns from $S \x T$ are permuted within
            $M_1$ or $M_2$, they will not change the $triangle-rank$. As such,
            there is no reason to permute rows indexed by $S$ in $M_2$ and no
            reason to permute columns indexed by $T$ in $M_1$. This creates a
            convenient ``mask'' that lets us permute rows and columns of $M$
            and build desired square submatrices inside $M_1$ and $M_2$ without 
            undoing any progress.
            \\\\
            Now we can put the two points from above together to define a
            procedure that proves the desired inequality. We start off with
            $M$. Next we permute rows and columns of $M$ to build the largest
            square submatrix with ones in diagonals on zeros below the diagonal
            in both $M_1$ and $M_2$ (showing $triangle-rank$ of these
            matrices). As discussed previously, we can do that without any
            clashes.
            \\\\
            Denote such square submatrices $S_1$ in $M_1$ and $S_2$ in $M_2$.
            Finally, we permute columns of resultant $M$ to put $S_1$ into the
            top left corner of block $S \x Y_2$, and then permute rows of
            resultant $M$ to put $S_2$ into the bottom right corner of block
            $X_1 \x T$. As a result of this operation, we have transformed $M$
            in such a way that it now contains a square matrix that is a tensor
            product of $S_1$ and $S_2$ (with the exception of the top right
            corner which contains arbitrary values and doesn't affect
            $triangle-rank$). Hence we have shown that if matrices $M_1$ and
            $M_2$ have $triangle-rank$ of $m$ and $n$ respectively, we can
            permute rows and columns of $M$ to build a square submatrix of side
            length $m + n$ with ones on the diagonal and zeroes below the
            diagonal, showing that the $triangle-rank$ of matrix $M$ must be at
            least $m + n$.

    \end{enumerate}

    \item We can assume $X \x Y$ is a square matrix - if it isn't, we can just
        pad out one of the sides to make it square. 
        \\\\
        \TODO{Can show this by proving that there must a big monochromatic
        rectangle of size $\geq 2^{-c(r)}|X||Y|$, then showing that we can
        recursively say yes or no to recurse into one of the smaller
        rectangles, at every step either halving the rank of making the matrix
        much smaller}.

    \item 

\end{enumerate}


\end{document}
