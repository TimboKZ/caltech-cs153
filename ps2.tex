%-- Generated using Rammy v0.1.0 | github.com/TimboKZ/Rammy
%-- Module: latex-common
%-- Template: problem-set

%-- summary: Template for lecture notes with a lot of useful snippets.

\documentclass{article}

\usepackage[table]{xcolor}
\usepackage{mathtools}

% Title for the `compact-header` snippet
\newcommand\HeaderSubject{CS 153~~~Communication Complexity}
\newcommand\HeaderAuthor{Timur Kuzhagaliyev}

%-- Rammy start ----------------
\input{latex-common/snippets/symbols.tex} %-- module-snippet: latex-common/symbols
\input{latex-common/snippets/problem-set-header.tex} %-- module-snippet: latex-common/problem-set-header
\input{latex-common/snippets/urls.tex} %-- module-snippet: latex-common/urls
\input{latex-common/snippets/figures.tex} %-- module-snippet: latex-common/figures
\input{latex-common/snippets/code.tex} %-- module-snippet: latex-common/code
\input{latex-common/snippets/misc.tex} %-- module-snippet: latex-common/misc
%-- Rammy end ------------------

% Don't wrap long matrices
\setcounter{MaxMatrixCols}{20}

\newcommand\tr{\textrm{triangle-rank}}

\begin{document}
\problemset{Problem Set \#2 Solutions}{Out: May 22}{Due: \bf {June 1}}

\begin{enumerate}
    \item
    \begin{enumerate}
        \item We can show that the inequality in the question holds by
            considering the structure of $M_1$ and $M_2$. The breakdown of
            matrix of $M$ into $M_1$ and $M_2$looks like
            \begin{gather*}
                \left(\begin{array}{c>{\columncolor{olive!20}}cc}
                \x  & X_1 \x T & \x \\
               \rowcolor{blue!20}
               S \x Y_1 & S \x T & S \x Y_2 \\
                \x & X_2 \x T & \x \\
              \end{array}\right)
            \end{gather*}

            where the highlighted row is matrix $M_1$, the highlighted column
            is the matrix $M_2$ and $X_1, S, X_2$ with $Y_1, T, Y_2$ are
            partitions of $X$ and $Y$ respectively. Note that due to the
            structure of given matrices, the standard rank of $M_1$ is bound
            above by $|S|$ and the standard rank of $M_2$ is bound above by
            $|T|$, and the same bounds apply to triangle-rank.
            \\\\
            Note also that $S \x T$ is a zero matrix - this means that
            regardless of how rows or columns from $S \x T$ are permuted within
            $M_1$ or $M_2$, they will not change the triangle-rank. As such,
            there is no reason to permute rows indexed by $S$ in $M_2$ and no
            reason to permute columns indexed by $T$ in $M_1$. This creates a
            convenient ``mask'' that lets us permute rows and columns of $M$
            and build desired square submatrices inside $M_1$ and $M_2$ without 
            undoing any progress.
            \\\\
            Now we can put the two points from above together to define a
            procedure that proves the desired inequality. We start off with
            $M$. Next we permute rows and columns of $M$ to build the largest
            square submatrix with ones in diagonals on zeros below the diagonal
            in both $M_1$ and $M_2$ (showing triangle-rank of these
            matrices). As discussed previously, we can do that without any
            clashes.
            \\\\
            Denote such square submatrices $S_1$ in $M_1$ and $S_2$ in $M_2$.
            Finally, we permute columns of resultant $M$ to put $S_1$ into the
            top left corner of block $S \x Y_2$, and then permute rows of
            resultant $M$ to put $S_2$ into the bottom right corner of block
            $X_1 \x T$. As a result of this operation, we have transformed $M$
            in such a way that it now contains a square matrix that is a tensor
            product of $S_1$ and $S_2$ (with the exception of the top right
            corner which contains arbitrary values and doesn't affect
            triangle-rank). Hence we have shown that if matrices $M_1$ and
            $M_2$ have triangle-rank of $m$ and $n$ respectively, we can
            permute rows and columns of $M$ to build a square submatrix of side
            length $m + n$ with ones on the diagonal and zeroes below the
            diagonal, showing that the triangle-rank of matrix $M$ must be at
            least $m + n$.
            \\

        \item We can prove the desired bound using induction on triangle-rank.
            We begin by outling a protocol that uses $N^0(M_f) + 1$ bits per
            round of communication. Note that the smallest non-disjoint
            0-monochromatic cover of $M_f$ contains $2^{N^0(M_f)}$ rectangles.
            Note that if we enumerate these rectangles from $1$ to
            $2^{N^0(M_f)}$, it will take exactly $N^0(M_f)$ bits to communicate
            the index of the rectangle.
            \\\\
            Now our protocol proceeds as follows. Alice and Bob maintain sets
            $A$ and $B$ respectively. $A$ indicates the set of \textit{columns}
            in the search space (for Alice). $B$ indicates the set of
            \textit{rows} in the search space (for Bob). Initially, both $A$
            and $B$ contain the entire matrix $M_f$.
            \\\\
            In the beginning, Alice holds some row $x$ and Bob holds some
            column $y$ in $M_f$ (as the inputs).  Alice begins iterating over
            all rectangles $R_A = S \x T$ in the 0-cover we fixed earlier such
            that:
            \begin{enumerate}
                \item $R_A$ intersects her row $x$,
                \item Restricting the matrix to columns indexed by $S$ cuts the
                    triangle-rank of the matrix by at least a factor of 2, and
                \item At least one of the columns of $R_A$ overlaps with set
                    $A$.
            \end{enumerate}
            \\\\
            All rectangles that do not satisfy above conditions are ignored.
            For every $R_A$, Alice transmits the index of $R_A$ to Bob. Bob
            replies with either $0$ or $1$ to indicate whether his column $y$
            also intersects $R_A$. If Bob returns $1$, then $x$ and $y$ both
            intersect the same 0-rectangle, and the output $f(x, y)$ must be 0,
            so we're done. Otherwise, Alice removes columns covered by $R_A$
            from her set $A$ and proceeds to the next loop iteration. The
            amount of bits exchanged per iteration is $N^0(M_f) + 1$.
            \\\\
            Eventually, Alice and Bob either agree on some rectangle that Alice
            proposed or Alice runs out of rectangles that satisfy the three
            conditions we specified. If Alice runs out of 0-rectangle
            candidates, she signals this to Bob using 1 bit. Assume that the
            actual output of $f(x,y)$ is $0$. In this case, there must exist a
            0-monochromatic rectangle $S \x T$ that intersects both row $x$ and
            column $y$. Denote the triangle-rank of submatrix $S \x Y$ by
            $\Delta M_1$ and triangle-rank of submatrix $X \x T$ by $\Delta
            M_2$. Since Alice cannot find the desired 0-rectangle, it must be
            the case that $\Delta M_1 > \frac{1}{2} \Delta M_f$. Combining this
            with the inequality from part (a), we get $\Delta M_2 < \frac{1}{2}
            \Delta M_f$. This means that Bob can run the same loop using his
            $x$ and $B$ to find the desired 0-rectangle and conclude that the
            output of $f(x, y)$ is indeed $0$, exchanging the same amount of
            bits per iteration.
            \\\\
            In the case when $f(x, y) = 1$, Bob will still attempt to execute
            the same loop, but he will eventually run out of candidate
            0-rectangles (since Alice will just reject everything). If this
            happens, Bob communicates this to Alice with one bit and the
            parties will conclude that the output is $1$.
            \\\\
            We can now use the protocol outline to prove the desired bound on
            communication complexity using induction on triangle-rank.
            \begin{itemize}
                \item \textbf{Inductive hypothesis}: Assume that the formula
                    $$D(M_f) \leq (\log_2(\tr(M_f) + 1) + 1)
                    (N^0(M_f) + 1)$$ holds for all $M_f = A' \x B'$ such that 
                    $\tr(A' \x B') < t$.

                \item \textbf{Base case}: $\tr(M_f) = 0$. In this case, we have
                    an all $0$ matrix, so there is a 0-cover of size 1, hence
                    $N^0(M_f) = \log_2(1) = 0$. Using our protocol, Alice
                    transmits $N^0(M_f)$ bits, or, in other words, doesn't
                    transmit anything since the index is trivial. Bob replies
                    with 0 to acknowledge that he's in the same rectangle. In
                    total, we have 1 bit of communication, and $D(M_f) = 1$.
                    Indeed, the formula holds for our base case:
                    \begin{align*}
                        D(M_f) &\leq (log_2(\tr(M_f) + 1) +1) (N^0(M_f) + 1)
                            \\ &= (log_2(0 + 1) +1) (0 + 1)
                            \\ &= (0+1) (1)
                            \\ &= 1
                    \end{align*}
                \item \textbf{Inductive step}: Need to show that formula holds
                    for $M_f = A \x B$ s.t. $\tr(A \x B) = t$. As per the
                    protocol we outlined above, Alice starts searching for a
                    suitable 0-rectangle from the cover first.
                    \\\\
                    \textbf{Case 1}: Alice finds some $S \x T$ rectangle $R_A$
                    and sends its index over to Bob. If Bob replies with $1$,
                    parties have found a common 0-rectangle and the protocol
                    can terminate with output $0$ after having exchanged only
                    $N^0(M_f) + 1$ bits. Clearly, the formula is satisfied.
                    \\\\
                    On the other hand, if Bob rejects $R_A$, Alice can remove
                    the columns that touch $R_A$ from her set $A$. We can show
                    that this action decreases the triangle-rank of her search
                    space by a factor of 2: For Alice to pick $R_A$,
                    restricting $M_f$ to the rows indexed by $S$ must've cut
                    the triangle-rank but 

                    \textbf{Case 2}: Alice doesn't find a suitable rectangle,
                    so she has to tell Bob that he needs to begin the search.
                    There are two possibilities here: If $f(x,y) = 0$, 
            \end{itemize}

    \end{enumerate}

    \item We can prove that the log-rank conjecture holds for our notion of
        rank by reducing the problem of calculating the output $f(x, y)$ to an
        instance of $CIS_G$. I will assume that matrix multiplication is done
        using the standard dot product of rows and columns (as opposed to
        $\inner \mod 2$).
        \\\\
        Denote the Boolean-rank of an $N \x N$ matrix $M$ as $r$. Then there
        must exist an $N \x r$ Boolean matrix $U$ and an $r \x N$ Boolean
        matrix $V$ such that $M = UV$. Note that if we use standard matrix
        multiplication, this is only possible when $\sum_{q=0}^r U_{iq} *
        V_{qj} \in \{0,1\}$ for all $i,j \in \{ 1, \ldots, N \}$. This, in
        turn, implies that every row $x$ in $U$ and every row $y$ in $V$ share
        at most one index $p$ such that $x_p * y_p = 1$ (for all remaining
        indices $p'$, it must be the case that $x_{p'} * y_{p'} = 0$). We can
        exploit this fact in our reduction to $CIS_G$.
        \\\\
        Given a function $f: X \x Y \rightarrow \{ 0,1 \}$ and the
        corresponding $M_f$ with $\textrm{Booleank-rank}(M_f) = r$, we can
        construct a graph $G$ as follows. $G$ will have $r$ nodes, denoted
        $a_i$ for $i \in \{ 1, \ldots, r \}.$ The node $a_i$ corresponds to
        $i$th column of $U$ and $i$th row of $V$. We then add edges as follows.
        For every row $x$ in $U$, we treat $x$ as a bitmap of nodes from $G$,
        where $x_i = 1$ means that the bitmap includes node $a_i$.  We connect
        all nodes $a_i$ that appear in the bitmap $x$ together, so that they
        form a clique. Graph $G$ is now ready to be used in $CIS_G$.
        \\\\
        The protocol works as follows. Before communiction begins, both Alice
        and Bob know $G$, $U$ and $V$. On input $x^*$ that corresponds to row
        $x$ in $M_f$, Alice looks at row $x$ in $U$ and treats it as a bitmap
        of nodes from $G$ that she has to choose as her clique, $C$. By
        defintion of $G$, her choice of $C$ is guaranteed to be a clique.
        \\\\
        On input $y^*$ that corresponds to column $y$ in $M_f$, Bob looks at
        column $y$ of $V$ and treats it as a bitmap of nodes from $G$ that he
        has to choose as his independent set, $I$. We can show by contradiction
        that Bob's choice of $I$ must be an independent set. Assume that $I$ is
        not an independent set. This means that there is an edge between some
        nodes in $I$. By construciton of $G$, this in turn implies that matrix
        $U$ has some row $y^+$ that shares more than one index $i$ with $x$
        such that $y^+_i * x_i = 1$, which implies that not all matrices in
        relationship $M = UV$ are Boolean matrices, which is a contradiction.
        Hence $I$ must be an indepdenent set.
        \\\\
        Note that no communication has yet occurred. Alice and Bob now solve
        problem $CIS_G$ using inputs $C$ and $I$, using output of $CIS_G(C, I)$
        as the result of running $f(x^*,y^*)$. Proof of correctness is trivial:
        if $CIS_G$ outputs 1, then $C$ and $I$ share some common node $a_i$.
        This implies that $x_i * y_i = 1$, $\inr{U_x, V^y} = 1$. The last
        expression is equivalent to caculating the value of the element in
        $M_f$ on which row $x$ and column $y$ intersect each other, i.e. the
        output of $f(x, y)$. Similarly, if $CIS_G$ outputs $0$, we have
        $\inr{U_x, V^y} = 0$ and $f(x,y) = 0$.
        \\\\
        We have shown in class that communication complexity of $CIS_G$ is
        bounded above by $O(\log^2 n)$, where $n$ is the size of the graph. In
        our case, size of the graph is $r = \textrm{Boolean-rank}(M_f)$, so we
        can say that $$ D(M_f) \leq O(\log^2 \textrm{Boolean-rank}(M_f)) $$
        This proves that log-rank conjecture holds for the proposed notion of
        rank.
        \\

    \item 
    \begin{enumerate}
        \item First, we can observe that the definition of discrepancy implies
            that we can invert Boolean outputs of the function without changing
            the discrepancy, since we're using the absolute difference between
            probabilities.
            \\\\
            We can start by looking at the full $X_1 \x \ldots \x X_k$ cylinder
            first. With this choice of $S$, the calculation for
            $$\abs{\Pr_\mu[f(x_1, \ldots, x_k) = 0 \land (x_1, \ldots x_k) \in
            S] - \Pr_\mu[f(x_1, \ldots, x_k) = 1 \land (x_1, \ldots x_k) \in S]
            }$$
            reduces into just
            $$\abs{ \Pr_\mu[f(x_1, \ldots, x_k) = 0] - \Pr_\mu[f(x_1, \ldots,
            x_k) = 1] }.$$

            It's clear that the maximum possible value is 1, when the function
            either outputs always 1 or always 0. When we don't have this
            trivial case, the two terms begin to cancel each other out.
            That is, if we WLOG assume $\Pr_\mu[f(\ldots) = 1]$ is the largest
            of the two, we know that $$\abs{ \Pr_\mu[f(\ldots) = 1] } > \abs{
            \Pr_\mu[f(\ldots) = 0] - \Pr_\mu[f(\ldots) = 1] }.$$ Due to this
            fact, we can aim to maximise the value by somehow getting rid of
            the smaller component. We can rewrite the definition of discrepancy
            as follows:
            $$ \textrm{disc}_\mu(f) = \max_S\; \abs{ \Pr_\mu \left(f(x^k) =
            1\right) \Pr_\mu \left(x^k \in S \given[\big] f(x^k) = 1\right) -
            \Pr_\mu \left(f(x^k) = 0\right) \Pr_\mu \left(x^k \in S
            \given[\big] f(x^k) = 0\right) }$$

            From now on, we will WLOG focus on 1-monochromatic cylinder
            intersections and assume given $f$ has nondeterministic complexity
            $t$. Identical argument applies to the co-nondeterminstic version.
            If we WLOG consider only 1-monochromatic cylinder intersections,
            it's clear the second term becomes zero and we're left off with
            just $\Pr_\mu \left(f(x^k) = 1\right) \Pr_\mu \left(x^k \in S
            \given[\big] f(x^k) = 1\right)$.
            \\\\
            We're given that $f$ has nondeterministic complexity $t$, which
            implies that there exists a non-disjoint cover of $f$ with
            1-monochromatic cylinder intersections, with the size of this cover
            being $2^t$. Note that because this cover can be non-disjoint,
            total sum of probabilities of an element belonging to these
            cylinders can be greater than 1. Since these cylinder intersections
            cover \textit{all} 1s and there are $2^t$ of them, they must
            include at least one cylinder intersection $S$ such that
            $\Pr_\mu(x^k \in S) \geq 2^{-t}$.  Otherwise, we would
            have $\Pr_\mu(x^k \in S) < 2^{-t}$ for all of the $2^t$ 1-cylinder
            intersections in the cover, which is a contradiction since the sum
            of probabilities will be less than 1.  As a result, we can conclude
            that there exists a 1-monochromatic cylinder intersection $S$ such
            that $\Pr_\mu \left(x^k \in S \given[\big] f(x^k) = 1\right) \geq
            2^{-t}$.
            \\\\
            The function $f$ and given distribution $\mu$ are fixed, so we can
            conclude that $$\Omega(\Pr_\mu \left(f(x^k) = 1\right) \Pr_\mu
            \left(x^k \in S \given[\big] f(x^k) = 1\right))$$ simplifies to
            just $\Omega(\Pr_\mu\left(x^k \in S \given[\big] f(x^k) =
            1\right))$, and, finally, using the result from above, to
            $\Omega(2^{-t})$ when we use the largest 1-monochromatic cylinder
            intersection. This shows that the discrepancy is bounded below by
            $\Omega(2^{-t})$ given that nondeterministic or co-nondeterministic
            communication complexity of $f$ is $t$.
            \\

        \item We can view the input for the problem DISJ$_k$ as a $k \x n$
            Boolean matrix, where each player $i$ holds the string
            corresponding to $i$th row. In this formulation, the co-DISJ$_k$
            problem reduces to finding a column of all 1's (which implies
            a global intersection).
            \\\\
            An all powerful prover can point out the index of the column that
            contains all 1's, which would require $\log n$ bits to represent.
            Then, the first player can check everyone else's values in that
            column and output 1 if they are indeed all 1's. The second player
            can check first player's value and output 1 in the same case,
            requring $O(\log n)$ bits in total to solve co-DISJ$_k$.
    \end{enumerate}

\end{enumerate}


\end{document}
