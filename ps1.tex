%-- Generated using Rammy v0.1.0 | github.com/TimboKZ/Rammy
%-- Module: latex-common
%-- Template: lecture-notes

%-- summary: Template for lecture notes with a lot of useful snippets.

\documentclass{article}

% Title for the `compact-header` snippet
\newcommand\HeaderTitle{\textbf{CS 153 Problem Set \#1 Solutions} by Timur Kuzhagaliyev}

%-- Rammy start ----------------
\input{latex-common/snippets/symbols.tex} %-- module-snippet: latex-common/symbols
\input{latex-common/snippets/compact-header.tex} %-- module-snippet: latex-common/compact-header
\input{latex-common/snippets/urls.tex} %-- module-snippet: latex-common/urls
\input{latex-common/snippets/figures.tex} %-- module-snippet: latex-common/figures
\input{latex-common/snippets/code.tex} %-- module-snippet: latex-common/code
%-- Rammy end ------------------

% Don't wrap long matrices
\setcounter{MaxMatrixCols}{20}

\usepackage{enumitem} 

\begin{document}


\begin{enumerate}

    \item Solutions:
        \begin{enumerate}[label=(\alph*)]
            \item Test
        \end{enumerate}

    \item TRIBES function requires that $x$ and $y$ share at least one 1 on
        every row of the matrix. Solutions:
        \begin{enumerate}[label=(\alph*)]
            \item The most straightforward way to solve $\textrm{TRIBES}(x, y)$
                is for Alice to just send over her entire matrix, which
                requires $n$ bits, and wait for Bob's answer, giving overall
                communication complexity $O(n)$. This proves the upper bound
                for the problem.
                
                TODO: Show lower bound.

            \item By definition of the problem, to show that
                $\textrm{TRIBES}(x,y) = 1$ we only need to find one index $j$
                on every row $i$ such that $x_{i,j} = y_{i,j} = 1$. Therefore
                an all-powerful prover just needs to point out these indices on
                each row. There are $\sqrt{n}$ rows, and each column index
                takes $\textrm{log}_2 (\sqrt{n})$ bits to represent, giving us
                an overall bit count of $\sqrt{n} \cdot \frac{1}{2}
                \textrm{log}_2 n$. Alice can send over these values to Bob and
                wait for him to confirm that there is a match. Hence the upper
                bound on $N^1(\textrm{TRIBES})$ is $O(\sqrt{n}\;\textrm{log}\,
                n)$.
                \\\\
                We can use the fooling set technique to show the lower bound.
                Consider a $1\x\sqrt{n}$ vector $e_i^T$, which has zeros
                everywhere except the $i$th position. Consider a matrix $x$
                where each row is a vector $e_{i_k}$ for some $i_k \in \{ 1,
                \ldots, \sqrt{n} \}$. Clearly, $\textrm{TRIBES}(x, x) = 1$
                because $x$ has at least one 1 on each row. At the same time,
                if we permute any of the rows of $x$ by shifting the 1 in that
                row left or right and define the new matrix $x^*$, we'll see
                that $\textrm{TRIBES}(x, x^*) = 0$ because there is now at
                least one row where $x$ and $x^*$ do not match up. We can
                exploit this to generate a fooling set. Let $M$ be the set of
                matrices that start off as an identity matrix, but have either
                1 within some row shifted left or right, or have some rows
                swapped, or both. Then we can define the fooling set $S$ as:
                $$ S = \{ ((M^*, M^*): \textrm{distinct } M^* \in M \} $$

                Clearly, this a 1-fooling set because for any two distinct
                pairs $(M_1, M_2), (M_1^*, M_2^*) \in S$, neither $(M_1,
                M_2^*)$ nor $(M_1^*, M_2)$ can be evaluated to 1 since they
                must differ in at least one place (we can show by contradiction
                that if this is not the case, the pairs must be identical). Now
                for the size of this fooling set: there are $\sqrt{n}$ rows in
                total, and for each row we have $\sqrt{n}$ different choices
                for the vector $e_i^T$. This means that $|S| =
                \sqrt{n}^{\sqrt{n}}$, and we can obtain a lower bound
                $N^1(\textrm{TRIBES}) \geq \Omega(\textrm{log}\, \left(
                \sqrt{n}^{\sqrt{n}} \right) )$, or, equivalently,
                $N^1(\textrm{TRIBES}) \geq \Omega(\sqrt{n}\;\textrm{log}\,n)$.
                \\\\
                Combining the lower and upper bound, we get
                $N^1(\textrm{TRIBES}) = \Theta(\sqrt{n}\;\textrm{log}\,n)$.
                \\

            \item To show that $\textrm{TRIBES}(x,y) = 0$, we need to find a
                single row $i$ where in each position $j$ we have $x_{i,j}
                \land y_{i,j} = 0$. Our all-powerful prover can identify this
                row and show its index to Alice and Bob. Then, Alice can send
                Bob a bitmask of said row, taking up $\sqrt{n}$ bits in total.
                Including the 1 bit of Bob's reply, we get the overall
                complexity of $O(\sqrt{n})$, proving the upper bound for
                $N^0(\textrm{TRIBES})$.
                \\\\
                The lower bound for $N^0$ can be proved by showing a 0-fooling
                set of size $2^{\sqrt{n}}$.
                \\
        \end{enumerate}

    \item Solutions:
        \begin{enumerate}[label=(\alph*)]
            \item A disconnected graph with a clique of size two and any amount
                of independent vertices. This way, whatever the size of
                independent set that Alice holds, she only ever needs
                communicate whether she holds one of the vertices in the
                clique. If we enumerate all nodes in $G$ as $1, \ldots, n$, the
                amount of bits exchanged is $2\, \textrm{log}\,n + 2$ (checking
                each of the nodes in the clique and waiting for Bob's reply),
                which gives us complexity $O(\textrm{log}\,n)$ overall.

            \item `
        \end{enumerate}

\end{enumerate}


\end{document}
